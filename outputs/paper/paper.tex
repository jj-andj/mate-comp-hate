% Options for packages loaded elsewhere
\PassOptionsToPackage{unicode}{hyperref}
\PassOptionsToPackage{hyphens}{url}
\PassOptionsToPackage{dvipsnames,svgnames,x11names}{xcolor}
%
\documentclass[
]{article}

\usepackage{amsmath,amssymb}
\usepackage{lmodern}
\usepackage{iftex}
\ifPDFTeX
  \usepackage[T1]{fontenc}
  \usepackage[utf8]{inputenc}
  \usepackage{textcomp} % provide euro and other symbols
\else % if luatex or xetex
  \usepackage{unicode-math}
  \defaultfontfeatures{Scale=MatchLowercase}
  \defaultfontfeatures[\rmfamily]{Ligatures=TeX,Scale=1}
\fi
% Use upquote if available, for straight quotes in verbatim environments
\IfFileExists{upquote.sty}{\usepackage{upquote}}{}
\IfFileExists{microtype.sty}{% use microtype if available
  \usepackage[]{microtype}
  \UseMicrotypeSet[protrusion]{basicmath} % disable protrusion for tt fonts
}{}
\makeatletter
\@ifundefined{KOMAClassName}{% if non-KOMA class
  \IfFileExists{parskip.sty}{%
    \usepackage{parskip}
  }{% else
    \setlength{\parindent}{0pt}
    \setlength{\parskip}{6pt plus 2pt minus 1pt}}
}{% if KOMA class
  \KOMAoptions{parskip=half}}
\makeatother
\usepackage{xcolor}
\usepackage[margin = 1in]{geometry}
\setlength{\emergencystretch}{3em} % prevent overfull lines
\setcounter{secnumdepth}{5}
% Make \paragraph and \subparagraph free-standing
\ifx\paragraph\undefined\else
  \let\oldparagraph\paragraph
  \renewcommand{\paragraph}[1]{\oldparagraph{#1}\mbox{}}
\fi
\ifx\subparagraph\undefined\else
  \let\oldsubparagraph\subparagraph
  \renewcommand{\subparagraph}[1]{\oldsubparagraph{#1}\mbox{}}
\fi


\providecommand{\tightlist}{%
  \setlength{\itemsep}{0pt}\setlength{\parskip}{0pt}}\usepackage{longtable,booktabs,array}
\usepackage{calc} % for calculating minipage widths
% Correct order of tables after \paragraph or \subparagraph
\usepackage{etoolbox}
\makeatletter
\patchcmd\longtable{\par}{\if@noskipsec\mbox{}\fi\par}{}{}
\makeatother
% Allow footnotes in longtable head/foot
\IfFileExists{footnotehyper.sty}{\usepackage{footnotehyper}}{\usepackage{footnote}}
\makesavenoteenv{longtable}
\usepackage{graphicx}
\makeatletter
\def\maxwidth{\ifdim\Gin@nat@width>\linewidth\linewidth\else\Gin@nat@width\fi}
\def\maxheight{\ifdim\Gin@nat@height>\textheight\textheight\else\Gin@nat@height\fi}
\makeatother
% Scale images if necessary, so that they will not overflow the page
% margins by default, and it is still possible to overwrite the defaults
% using explicit options in \includegraphics[width, height, ...]{}
\setkeys{Gin}{width=\maxwidth,height=\maxheight,keepaspectratio}
% Set default figure placement to htbp
\makeatletter
\def\fps@figure{htbp}
\makeatother
\newlength{\cslhangindent}
\setlength{\cslhangindent}{1.5em}
\newlength{\csllabelwidth}
\setlength{\csllabelwidth}{3em}
\newlength{\cslentryspacingunit} % times entry-spacing
\setlength{\cslentryspacingunit}{\parskip}
\newenvironment{CSLReferences}[2] % #1 hanging-ident, #2 entry spacing
 {% don't indent paragraphs
  \setlength{\parindent}{0pt}
  % turn on hanging indent if param 1 is 1
  \ifodd #1
  \let\oldpar\par
  \def\par{\hangindent=\cslhangindent\oldpar}
  \fi
  % set entry spacing
  \setlength{\parskip}{#2\cslentryspacingunit}
 }%
 {}
\usepackage{calc}
\newcommand{\CSLBlock}[1]{#1\hfill\break}
\newcommand{\CSLLeftMargin}[1]{\parbox[t]{\csllabelwidth}{#1}}
\newcommand{\CSLRightInline}[1]{\parbox[t]{\linewidth - \csllabelwidth}{#1}\break}
\newcommand{\CSLIndent}[1]{\hspace{\cslhangindent}#1}

\definecolor{ltgray}{HTML}{EFEFEF}
\makeatletter
\makeatother
\makeatletter
\makeatother
\makeatletter
\@ifpackageloaded{caption}{}{\usepackage{caption}}
\AtBeginDocument{%
\ifdefined\contentsname
  \renewcommand*\contentsname{Table of contents}
\else
  \newcommand\contentsname{Table of contents}
\fi
\ifdefined\listfigurename
  \renewcommand*\listfigurename{List of Figures}
\else
  \newcommand\listfigurename{List of Figures}
\fi
\ifdefined\listtablename
  \renewcommand*\listtablename{List of Tables}
\else
  \newcommand\listtablename{List of Tables}
\fi
\ifdefined\figurename
  \renewcommand*\figurename{Figure}
\else
  \newcommand\figurename{Figure}
\fi
\ifdefined\tablename
  \renewcommand*\tablename{Table}
\else
  \newcommand\tablename{Table}
\fi
}
\@ifpackageloaded{float}{}{\usepackage{float}}
\floatstyle{ruled}
\@ifundefined{c@chapter}{\newfloat{codelisting}{h}{lop}}{\newfloat{codelisting}{h}{lop}[chapter]}
\floatname{codelisting}{Listing}
\newcommand*\listoflistings{\listof{codelisting}{List of Listings}}
\makeatother
\makeatletter
\@ifpackageloaded{caption}{}{\usepackage{caption}}
\@ifpackageloaded{subcaption}{}{\usepackage{subcaption}}
\makeatother
\makeatletter
\@ifpackageloaded{tcolorbox}{}{\usepackage[many]{tcolorbox}}
\makeatother
\makeatletter
\@ifundefined{shadecolor}{\definecolor{shadecolor}{rgb}{.97, .97, .97}}
\makeatother
\makeatletter
\makeatother
\ifLuaTeX
  \usepackage{selnolig}  % disable illegal ligatures
\fi
\IfFileExists{bookmark.sty}{\usepackage{bookmark}}{\usepackage{hyperref}}
\IfFileExists{xurl.sty}{\usepackage{xurl}}{} % add URL line breaks if available
\urlstyle{same} % disable monospaced font for URLs
\hypersetup{
  pdftitle={Harmful Anti-Foreign Sentiments based on Concern for Competition Should be Recognized and Addressed},
  pdfauthor={Jayden Jung, Finn Korol, Sofia Sellitto},
  colorlinks=true,
  linkcolor={blue},
  filecolor={Maroon},
  citecolor={Blue},
  urlcolor={Blue},
  pdfcreator={LaTeX via pandoc}}

\title{Harmful Anti-Foreign Sentiments based on Concern for Competition
Should be Recognized and Addressed\thanks{Code and data are available
at: https://github.com/jj-andj/mate-comp-hate ; Replication on Social
Science Reproduction platform available at:
https://doi.org/10.48152/ssrp-qg85-cb34}}
\author{Jayden Jung, Finn Korol, Sofia Sellitto}
\date{February 17, 2023}

\begin{document}
\maketitle
\begin{abstract}
Globalization, immigration, and asylum seeking is a common topic of
discussion, one which most people hold some personal opinion on
accompanied by certain justifications. This paper analyzes data on what
proportion of non-immigrant German men are likely to perceive refugees
as threats to finding a romantic partner relative to the male-to-female
ratio within their municipality and which of them support sentiments of
anti-refugee violence. Results show that there can be argued an effect
of these sentiments on actual rates of hate crime. We apply secondary
research regarding Canadian rates of immigration and gender imbalances
and raise concerns regarding the possibly generalizable nature of the
findings in Germany. As this issue specifically affects minority groups
experiencing prejudice and even furthers their marginalization, we place
great emphasis on the weight of this discussion and propose that it
should be considered to inform policies or initiatives intending to
address racism and hate crimes, especially in breaking down the framing
of refugees as a threat to non-immigrants, whether that's through
education, public messaging, or other implementations.
\end{abstract}
\ifdefined\Shaded\renewenvironment{Shaded}{\begin{tcolorbox}[breakable, frame hidden, interior hidden, enhanced, borderline west={3pt}{0pt}{shadecolor}, sharp corners, boxrule=0pt]}{\end{tcolorbox}}\fi

\hypertarget{introduction}{%
\section{Introduction}\label{introduction}}

In Canada, hate crimes based on race and ethnicity increased by 80\% in
2020, with the highest number of incidents targeting black individuals,
followed by east and southeast Asians, indigenous individuals, and the
lowest number of victims being South Asian individuals (Moreau and Wang
2022). Seeing that Canada is one of the most diverse countries in the
world, welcoming 405,999 permanent immigrants in 2021 and 130,125
refugees in 2020, these statistics concerning the increase of hate
crimes pose a real and visceral threat to a large proportion of Canadian
residents ({``2022 Annual Report to Parliament on Immigration''} 2022).

There are many factors that contribute to the increase in hate crimes in
Canada. However, with the rise of far-right discourse in the United
States, anti-immigrant and anti-refugee rhetoric is becoming more
prevalent in Canada. These negative attitudes and actions towards
immigrants, refugees, and marginalized individuals can be the result of
various structural and personal factors, including increased competition
in the job and housing markets, resource scarcity, misguided beliefs
about crime rates, illness, welfare dependency, and fears of losing
national identity. Despite this, one factor that has received little
attention until recently is the impact of competition in dating and
marriage markets. A 2021 paper by Dancygier, Egami, Jamal, and Rischke
published in the American Journal of Political Science delves into this
important and often-overlooked area of research (Dancygier et al. 2021).

In studying the opinions of German males living in municipalities with
excess male populations, they find that a portion of non-immigrant
German men hold the belief that refugees pose a threat to their ability
to pursue German women (Dancygier et al. 2021). Their findings for the
estimand of non-immigrant German men suggest that hate crimes increase
where non-immigrant German men are disadvantaged in their local dating
markets (Dancygier et al. 2021). By using ecological evidence and
originally curated survey data, the paper concludes that competition in
dating and marriage markets where men outnumber women increase
anti-refugee sentiments and violence (Dancygier et al. 2021).

Our paper will follow a reproduction of Dancygier, Egami, Jamal, and
Rischke's findings and apply a Canadian facing lens to discuss its
implications on local Canadian populations and increased anti-refugee/
immigrant sentiments and violence. In addressing the following two
research claims, (1) Non-immigrant German men who live in municipalities
with excess male populations are more likely to perceive refugees as
threats and (2) Non-immigrant German men who perceive mate competition
are more likely to support violence as the only means to gain the
attention of German politicians, we conclude that \_\_\_\_\_\_\_\_INSERT
CONCLUSION HERE\_\_\_\_\_.

We will first discuss \_\_\_\_\_\_\_\_INSERT FORMT
HERE\_\_\_\_\_\_\_\_\_\_

\hypertarget{running-code}{%
\section{Running Code}\label{running-code}}

When you click the \textbf{Render} button a document will be generated
that includes both content and the output of embedded code. You can
embed code like this:

\textless\textless\textless\textless\textless\textless\textless{} HEAD
\textless\textless\textless\textless\textless\textless\textless{} HEAD

\hypertarget{data}{%
\section{Data}\label{data}}

\hypertarget{source}{%
\subsection{Source}\label{source}}

The paper used for replication is from the American Journal of Political
Science which follows a discussion on the correlation of perceived mate
competition and its contributions to anti-refugee sentiments and higher
crime rates in Germany (cite paper). Our reproduction seeks to address
two claims made from the original paper and apply a Canadian facing
lens. The two claims are as follows: Are non-immigrant German men who
live in municipalities with excess male populations more likely to
perceive refugees as threats? Are non-immigrant German men who perceive
mate competition more likely to support violence as the only means to
gain the attention of german politicians? To collect this data, the
original paper uses four waves of online survey data collected in
Germany that are representative of gender, age and state (geographic
location) (Dancygier et al. 2021).

\hypertarget{methodology}{%
\subsection{Methodology}\label{methodology}}

This paper will replicate the survey data that was originally collected
for the 2021 paper by Dancygier, Egami, Jamal, and Rischkes, as
previously mentioned. Using the online survey platform ``Respondi'',
they conduct four waves of surveys representative which spanned from
September 2016 to December 2017 (Dancygier et al. 2021). The researchers
suggest that the anonymity provided by the online survey platform
resulted in respondents answering more truthfully. (Dancygier et al.
2021). To mitigate potential biases, the researchers employed list
experiments in Waves 1 and 2, and in Wave 2, they randomly assigned 50\%
of the sample to either a control or treatment group. The treatment
group was exposed to statements concerning their agreement with using
violence against refugees as a means to get the attention of German
politicians. However, no evidence was found to suggest that respondents
were concealing their support for hate crimes when comparing the means
of the control and treatment groups (Dancygier et al. 2021).

\hypertarget{features}{%
\subsection{Features}\label{features}}

\hypertarget{results}{%
\section{Results}\label{results}}

\hypertarget{discussion}{%
\section{Discussion}\label{discussion}}

\hypertarget{findings}{%
\subsection{Findings}\label{findings}}

\hypertarget{implications}{%
\subsection{Implications}\label{implications}}

\hypertarget{ethical-implications}{%
\subsection{Ethical Implications}\label{ethical-implications}}

In their paper, Dancygier, Egami, Jamal, and Rischke examine the ethical
implication of using experimental methodologies to investigate their
research topic. By using descriptive data in the form of surveys they
are able to investigate the opinions of non-immigrant German males and
their perception of mate competition and its translation to anti-refugee
violence. By avoiding experimental trials, they were able to explore
their topic without provoking anti-refugee sentiment, which might have
been a possible outcome had trials been conducted (Dancygier et al.
2021). While conducting experimental trials on this topic of research is
considered unethical, surveys and questionnaires may have a tendency to
give respondents the impression that their opinions are commonly shared
or even accurate. By being presented with a platform to express their
perception of mate competition and if they agree that violence towards
refugees is the only way to garner the attention of German politicians,
respondents may feel that their opinions are incorrectly justified. This
then has the potential to translate to violence towards German refugees.

\hypertarget{accounting-for-bias}{%
\subsection{Accounting for Bias}\label{accounting-for-bias}}

Ethical implications and biases arise naturally when collecting
quantitative and qualitative data. In their paper, Dancygier, Egami,
Jamal, and Rischke use online survey platforms to assess if Germans
living in areas with greater populations of men who experience turmoil
in the mating market are more likely to perceive competition between
themselves and refugees, moreover, does this ideology predict hate crime
support (Dancygier et al. 2021). The authors attempted to address
ethical concerns and statistical biases by utilizing control groups and
replicating their study with different samples and polling firms.
However, one potential bias that is challenging to control for is the
presence of sampling bias (Dancygier et al. 2021). Sampling bias occurs
when participants in a study are not representative of the estimand or
the ideal population of interest. One method to control for this bias is
simple random sampling, where participants are chosen by chance. Meaning
that every individual in the population/ estimand has an equal chance of
being selected. However, in their study, they were unable to utilize
simple random sampling. Instead attempted to make their survey results
representative by conducting four waves of surveys meant to be
representative of age, gender and state/ geographical region
{[}Dancygier et al. (2021)\}. Despite their effort, their survey results
may not be entirely representative as individuals who have a strong
interest in the subject matter are more likely to participate, meaning
they do not reflect the views of all non-immigrant German males
(Dancygier et al. 2021).

\hypertarget{limitations}{%
\subsection{Limitations}\label{limitations}}

\hypertarget{future-research}{%
\subsection{Future Research}\label{future-research}}

======= \#\#Figure 1 not including p value, percentage bar graph,
detailed explanation of tercile meanings ======= \#\#Figure 1 not
including p value, percentage bar graph, detailed explanation of tercile
meanings

\begin{quote}
\begin{quote}
\begin{quote}
\begin{quote}
\begin{quote}
\begin{quote}
\begin{quote}
f336ff80e09a5ca9d69c9226f7e483b791c5b1be
\end{quote}
\end{quote}
\end{quote}
\end{quote}
\end{quote}
\end{quote}
\end{quote}

\#\#Figure 2 add prevent bar

\#\# Table 1 have it coded in completely (check marks), better
description of meanings, labels of models (model1, model2,\ldots)

\begin{table}

\end{table}

\textless\textless\textless\textless\textless\textless\textless{} HEAD
\textgreater\textgreater\textgreater\textgreater\textgreater\textgreater\textgreater{}
067cdc6443e288bab4506987b7e9091b059a954e =======

\begin{quote}
\begin{quote}
\begin{quote}
\begin{quote}
\begin{quote}
\begin{quote}
\begin{quote}
f336ff80e09a5ca9d69c9226f7e483b791c5b1be \clearpage
\end{quote}
\end{quote}
\end{quote}
\end{quote}
\end{quote}
\end{quote}
\end{quote}

\hypertarget{references}{%
\section*{References}\label{references}}
\addcontentsline{toc}{section}{References}

\hypertarget{refs}{}
\begin{CSLReferences}{1}{0}
\leavevmode\vadjust pre{\hypertarget{ref-government_of_canada_2022}{}}%
{``2022 Annual Report to Parliament on Immigration.''} 2022.
\emph{Government of Canada}, November.
\url{https://www.canada.ca/en/immigration-refugees-citizenship/corporate/publications-manuals/annual-report-parliament-immigration-2022.html}.

\leavevmode\vadjust pre{\hypertarget{ref-main_paper}{}}%
Dancygier, Rafaela, Naoki Egami, Amaney Jamal, and Ramona Rischke. 2021.
{``Hate Crimes and Gender Imbalances: Fears over Mate Competition and
Violence Against Refugees.''} \emph{American Journal of Political
Science} 66 (2): 501--15. \url{https://doi.org/10.1111/ajps.12595}.

\leavevmode\vadjust pre{\hypertarget{ref-readstata13}{}}%
Garbuszus, Jan Marvin, and Sebastian Jeworutzki. 2021.
\emph{Readstata13: Import 'Stata' Data Files}.
\url{https://CRAN.R-project.org/package=readstata13}.

\leavevmode\vadjust pre{\hypertarget{ref-moreau_wang_2022}{}}%
Moreau, Gregory, and Jing Hui Wang. 2022. {``Police-Reported Hate Crime
in Canada, 2020.''} \emph{Government of Canada, Statistics Canada},
March.
\url{https://www150.statcan.gc.ca/n1/pub/85-002-x/2022001/article/00005-eng.htm}.

\leavevmode\vadjust pre{\hypertarget{ref-here}{}}%
Müller, Kirill. 2020. \emph{Here: A Simpler Way to Find Your Files}.
\url{https://CRAN.R-project.org/package=here}.

\leavevmode\vadjust pre{\hypertarget{ref-citeR}{}}%
R Core Team. 2020. \emph{R: A Language and Environment for Statistical
Computing}. Vienna, Austria: R Foundation for Statistical Computing.
\url{https://www.R-project.org/}.

\leavevmode\vadjust pre{\hypertarget{ref-pBrackets}{}}%
Schulz, Andreas. 2021. \emph{pBrackets: Plot Brackets}.
\url{https://CRAN.R-project.org/package=pBrackets}.

\leavevmode\vadjust pre{\hypertarget{ref-MASS}{}}%
Venables, W. N., and B. D. Ripley. 2002. \emph{Modern Applied Statistics
with s}. Fourth. New York: Springer.
\url{https://www.stats.ox.ac.uk/pub/MASS4/}.

\leavevmode\vadjust pre{\hypertarget{ref-tidyverse}{}}%
Wickham, Hadley, Mara Averick, Jennifer Bryan, Winston Chang, Lucy
D'Agostino McGowan, Romain François, Garrett Grolemund, et al. 2019.
{``Welcome to the {tidyverse}.''} \emph{Journal of Open Source Software}
4 (43): 1686. \url{https://doi.org/10.21105/joss.01686}.

\leavevmode\vadjust pre{\hypertarget{ref-readr}{}}%
Wickham, Hadley, Jim Hester, and Jennifer Bryan. 2023. \emph{Readr: Read
Rectangular Text Data}. \url{https://CRAN.R-project.org/package=readr}.

\leavevmode\vadjust pre{\hypertarget{ref-sandwich}{}}%
Zeileis, Achim. 2006. {``Object-Oriented Computation of Sandwich
Estimators.''} \emph{Journal of Statistical Software} 16 (9): 1--16.
\url{https://doi.org/10.18637/jss.v016.i09}.

\leavevmode\vadjust pre{\hypertarget{ref-lmtest}{}}%
Zeileis, Achim, and Torsten Hothorn. 2002. {``Diagnostic Checking in
Regression Relationships.''} \emph{R News} 2 (3): 7--10.
\url{https://CRAN.R-project.org/doc/Rnews/}.

\end{CSLReferences}



\end{document}
